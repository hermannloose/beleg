\section{Die LLVM Compilerarchitektur}

\begin{leftbar}
  \begin{itemize}
    \item Wie groß soll bzw. darf das ausfallen?
  \end{itemize}
\end{leftbar}

\subsection{Überblick}

\begin{leftbar}
  \begin{itemize}
    \item 1-2 Sätze, wer, wo, wann, was
    \item Frontends, IR, Backends, Tools beschreiben
    \item IR genauer ausführen, Basic Blocks vs. Funktionen
  \end{itemize}
\end{leftbar}

\todo[inline]{aufgrund der vielen Fachbegriffe insgesamt evtl. doch Englisch?}

\begin{leftbar}
\subsection{Analysis \& Transformations}

Platform and language agnostic analysis and transformations on LLVM bitcode and
assembly during optimization are implemented by so-called
\emph{Passes}~\cite{llvm-passes}. These exist at different granularities,
handling complete modules, individual functions, loop regions, basic blocks
etc. with corresponding constraints on which structures they can inspect or
modify.
\end{leftbar}

\subsection{Analyse \& Transformationen}

Analyse und Transformationen von LLVM Bitcode bzw. Assembly im Rahmen der
Optimierung werden sprach- und plattformunabhängig durch sogenannte
\emph{Passes}~\cite{llvm-passes} realisiert. Diese werden für bestimmte
Strukturebenen formuliert, betrachten z.B. komplette Module, einzelne
Funktionen, Schleifenregionen oder Basic Blocks und unterliegen entsprechenden
Beschränkungen. So sind im Allgemeinen Modifikationen außerhalb der aktuell
betrachteten Funktion etc.  verboten, um dem \emph{PassManager} eine möglichst
effiziente Anwendung der gewünschten Passes auf das gerade bearbeitete Modul zu
ermöglichen.

Zur Ausnutzung von örtlicher Lokalität wird dabei z.B. eine Menge von
\emph{FunctionPass}es zunächst der Reihe nach für eine einzelne Funktion
ausgeführt, bevor die nächste Funktion betrachtet wird, wobei keine Aussage
über die Folge der Funktionen getroffen wird.

Passes können Abhängigkeiten zu anderen Passes deklarieren, die z.B. Analysen
zum aktuell betrachteten Code liefern oder benötigte Vorbedingungen herstellen.
Zusätzlich müssen Passes explizit angeben, welche eventuell vorangegangenen
Passes ihre Gültigkeit \emph{behalten}—indem sie betroffene Analysewerte unter
Umständen selbst aktualisieren—da vom PassManager ansonsten vermeintlich
ungültig gewordene Analysen erneut durchgeführt werden, was bei Nichtbeachtung
zu Performanceeinbußen führen kann.

Die Realisierung eigener Analysen und Transformationen beschränkt sich damit
auf die Implementierung einer Unterklasse des gewünschten
Passtyps.~\cite{writing-passes}

\subsection{Besonderheiten im Bezug auf CFCSS}

CFCSS baut auf einem einzelnen, funktionsübergreifenden Kontrollflussgraphen
auf. Dabei terminieren sowohl Sprunginstruktionen als auch Funktionsaufrufe und
zugehörige \todo{ist dieser Sprachmischmasch legitim?}Returns einen Basic
Block. Kanten im CFG existieren jeweils zwischen Quelle und Ziel, somit auch
von Basic Blocks in der Mitte einer Funktion zum ersten Basic Block einer
anderen.

\texttt{call}-Instruktionen in LLVM stehen hingegen nicht am Ende eines Basic
Blocks, zugleich beschränkt sich der CFG jeweils auf die aktuelle Funktion. Als
Nachfolger eines Basic Blocks tauchen entsprechend nur die möglichen Ziele
seiner terminierenden Instruktion auf—neben Sprüngen auch in LLVM Assembly
nativ abgebildete \texttt{switch}-Statements—und keine Basic Blocks von
aufgerufenen Funktionen.
