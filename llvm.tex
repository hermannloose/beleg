\section{The LLVM compiler infrastructure}

\subsection{Overview}

\subsection{LLVM Assembly \& Bitcode}

\tobewritten[inline]{fill}

\subsection{Tools}

\tobewritten[inline]{fill}

\todo[inline]{figure out how much of this is needed}

\begin{description}
    \item[clang] \todo{fill}
    \item[opt] \todo{fill}
    \item[llvm-ld] \todo{fill}
\end{description}

\subsection{Analysis \& Transformations}

Platform and language agnostic analysis and transformations on LLVM bitcode and
assembly during optimization are implemented by so-called
\emph{Passes}~\cite{llvm-passes}. These exist at different granularities,
handling \needspolish{more general and explain later?}complete modules,
components of the callgraph, individual functions, loops, single entry/exit
regions and basic blocks with corresponding constraints on which structures
they can inspect or modify or whether state can be kept across invocations.

This allows the \emph{PassManager} to schedule the execution of multiple passes
on a module efficiently. Spatial locality is exploited for instance by first
running all requested \emph{FunctionPasses} on the same function before
proceeding to the next one.

\needspolish{schwurbelig?}While all passes except ModulePass and
CallGraphSCCPass do not have any guarantees about the order in which they are
handed functions, basic blocks etc. they can state dependencies on other passes
guaranteed to be run on the concerning part of the program before execution of
the requesting pass~\cite[Specifying interactions between
passes]{writing-passes}. This can be used to provide code analysis, usually
driving later modifications or to establish certain preconditions needed by
other passes.

Additionally passes have to explicitly state which of the previously executed
passes they \emph{do not} invalidate. In the absence of this information “all
passes are assumed to invalidate all others”~\cite[Specifying interactions
between passes]{writing-passes}. This may hurt performance when ignored, as the
PassManager will schedule invalidated analyses to be computed again once they
are needed by subsequent passes.

\todo[inline]{expand}

\subsection{Pitfalls in mapping CFCSS to LLVM}

CFCSS is based on a single CFG comprising all functions. Branching instructions
as well as function calls and returns are control flow transfers that terminate
a basic block. Consequently edges in the CFG may exist from basic blocks in the
middle of a function to the first basic block of another function.

Normal \texttt{call} instructions in LLVM however do not terminate basic
blocks~\cite{terminatorinst} \bjoern{evtl. lenkt das zu sehr ab
/ unwichtig?}(unlike \texttt{invoke} instructions, which might branch to an
error handler) and the called function does not appear in the set of successors
of the surrounding basic block, as each function in LLVM has its own separate
CFG.

\todo{check whether this still holds once work is finished} Upon starting this
work, \needspolish{holprig?}the above property suggested treating control flow
within a function separately from control flow occurring between functions.
Control flow inside a function can clearly be handled by a FunctionPass, while
control flow between functions might necessitate modifications in both
functions involved, thus requiring a CallGraphSCCPass or ModulePass.
