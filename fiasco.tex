\section{Übertragung auf Fiasco}

\subsection{Konfiguration \& Anpassung von Fiasco}

\paragraph{Inline Assembly}

Der in LLVM integrierte Assembler führte an mehreren Stellen zu Problemen. Auf
seine Verwendung wurde später verzichtet.

In \texttt{/boot/ia32/boot\_idt.S} verwendet Fiasco das GNU
Assembler-spezi\-fi\-sche Feature der
\emph{Sub-Sections}~\cite{gas-subsections}, welches von LLVM nicht unterstützt
wird. Auf Hinweis von Adam Lackorzyński wurde diese Unterteilung zunächst
aufgehoben, was aber im späteren Bootvorgang zu neuen Fehlern führte.

Wie in \cite[Inline assembly]{clang-compatibility} beschrieben behandelt der
LLVM Assembler uneindeutige x86-Instruktionen, welchen das die Bitbreite der
Operation bestimmende Suffix fehlt, als Fehler. Die Verwendung dieser
Instruktionen wurde ebenfalls an den entsprechenden Stellen korrigiert.

\todo{Problem genauer beschreiben}
Ein weiteres Problem trat bei der Makroexpansion in
\texttt{/kern/ia32/32\-/en\-try.S} und \texttt{/kern/ia32/32\-/entry-native.S} auf.

Nach einem Hinweis auf der Clang-Mailingliste wurde die Verwendung des
integrierten Assemblers danach mittels der Option
\texttt{-no-integrated-as}~\cite{manclang} vorerst ausgesetzt und auf den GNU
Assembler zurückgegriffen. Die in Assembler formulierten Teile von Fiasco
bedurften damit keiner Anpassung, die bisher erfolgten Änderungen wurden
rückgängig gemacht.
