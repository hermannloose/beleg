\section{Control-Flow Checking by Software Signatures}

\tobewritten[inline]{fill}

\subsection{Assumptions}

\todo{find a nicer subsection title}

\tobewritten[inline]{fill}

\paragraph{Control-Flow Graph}

CFCSS builds upon a \emph{control-flow graph} whose vertices are so-called
\emph{basic blocks}: code sequences with linear control flow, terminated by
branching instructions such as calls, returns and jumps, which in turn define
edges to other basic blocks in the graph.

\tobewritten[inline]{signature register GSR}
\tobewritten[inline]{static block signatures}

\paragraph{Branch fan-in nodes}

While the method described works well for linear control flow, what needs
special treatment are so-called \emph{branch fan-in nodes} that are the target
of multiple control-flow transfers.

\tobewritten[inline]{introduction of D}

\paragraph{Aliasing}

Using a runtime adjusting signature in that fashion however poses a subtle
problem for certain control-flow patterns. Take for example XYZ

\tobewritten[inline]{picture \& description}
\tobewritten[inline]{using Hamming-distance (paper)}
\tobewritten[inline]{using proxy blocks (my solution)}

\subsection{Existing measurements}

\tobewritten[inline]{fill}
