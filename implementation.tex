\section{Implementierung}

Für die Implementierung der \todo{checken, dass die dann auch genannt
wurden}genannten Passes wurde ein \emph{out-of-source} LLVM-Pro\-jekt außerhalb
des LLVM-Repository begonnen. Die so erstellten Objektdateien können von
\texttt{opt} und \texttt{llvm-ld} jeweils mittels der Option
\texttt{-load=\textit{object-file}} eingebunden und die enthaltenen Passes
verwendet werden.

\begin{leftbar}
  \begin{itemize}
    \item LLVM-Projekte \emph{out-of-source}
  \end{itemize}
\end{leftbar}

\subsection{Probleme bei der Verwendung von LLVM}

\begin{leftbar}
  \begin{itemize}
    \item LLVM ist ein \emph{optimierender} Compiler
    \item Annahme: Hardware arbeitet perfekt
    \item Constant Folding
    \item Dead Code Elimination
  \end{itemize}
\end{leftbar}
